\documentclass[rnd]{mas_proposal}
% \documentclass[thesis]{mas_proposal}

\usepackage[utf8]{inputenc}
\usepackage{amsmath}
\usepackage{amsfonts}
\usepackage{amssymb}
\usepackage{graphicx}

\title{Object detection in adverse weather conditions using tightly-coupled data-driven multi-modal sensor fusion}
\author{Kevin Patel}
\supervisors{Prof. Dr.-Ing. Sebastian Houben\\M.Sc. Santosh Thoduka}
\date{February 2023}

% \thirdpartylogo{path/to/your/image}

\begin{document}

\maketitle

\pagestyle{plain}

\section{Introduction}

\subsection{Topic of This R\&D Project}
\begin{itemize}
    \item What is Sensor Fusion?
    \begin{itemize}
        \item The process of combining data from multiple sensors to provide a more accurate, reliable, and comprehensive understanding of an environment or situation.
    \end{itemize}

    % Why do we need multi-modal sensor fusion?
        % Check where to put this question

    \item Fuse different sensor modalities like point cloud, pixels, and time series 
    \item Improve perception in adverse weather conditions e.g., fog, rain, snow, overcast, sleet, night
    \item Synchronization of multi-modal data
    \item Process dense and sparse resolution sensors data
    \item Make use of a data-driven approach
    \item Geometrically alignments of different sensors
    \item Real-time sensor fusion with low latency

    \item Topic naming convention
        \begin{itemize}
            \item {Object detection}
            \begin{itemize}
                \item 2D object detection - car, truck, pedestrian, cycle
            \end{itemize}
        
            \item {Adverse weather conditions}
            \begin{itemize}
                \item Fog, snow, rainy, overcast, sleet, dust
            \end{itemize}
        
            \item {Tightly-coupled}
            \begin{itemize}
                \item How different modalities are combined at what level
                \begin{itemize}
                    \item Eg. searly fusion, mid fusion/feature fusion, late fusion, ROI fusion, decision fusion
                \end{itemize}
            \end{itemize}
        
            \item {Data-driven}
            \begin{itemize}
                \item Using previously collected data or publicly available datasets
            \end{itemize}
        
            \item {Multi-modal}
            \begin{itemize}
                \item Using different data modalities
                \begin{itemize}
                    \item Sensors: Lidar, camera, IMU, GPS, infrared, radar
                    \item Datatypes: Point cloud, image, timer series
                \end{itemize}
            \end{itemize}
        
            \item {Sensor fusion}
            \begin{itemize}
                \item Fuse different sensors data to get a better estimation of an environment
            \end{itemize}
        \end{itemize}

% *************************
    % \item Provide reasonably detailed description of what you intent to do in your R\&D project.
    % \item You may also discuss the challenges that you have to address.
    % \item Reflect on the profile of the reader and PLEAAAASE, tell a story here and refrain from bombarding the readers with details which they may not be able to appreciate.
% **************************
\end{itemize}

\subsection{Relevance of This R\&D Project}
\begin{itemize}
% CAGR - Compound Annual Growth Rate

    \item According to Federal Highway Administration(FHA), adverse weather-related vehicle crashes cause over 5,000 fatalities and over 418,000 injuries each year in the United States. 
    \cite{federal-highway-administration-no-date}

    % Insurance Institute for Highway Safety (IIHS)
    \item The IIHS also found that in snowy weather, the fatal crash rate is 21\% higher than on clear roads, while during sleet and freezing rain, the rate is even higher at 37\%.

    \item According to the European Commission, 25\% of all road accidents in Europe happen due to adverse weather conditions, from highest to lowest: frost and ice, snow, and rain.
    \cite{cookson-2022}

    \item Autonomous vehicles: according to Marketsandmarkets, the sensor fusion market for autonomous vehicles is expected to reach \$ 22.2 billion by 2030 at a CAGR of 25.4\%  
    \cite{marketsandmarkets}

    \item According to the Federal Highway Administration (FHWA), poor visibility is a contributing factor in over 7,000 annual crashes in the United States.
    
    \item According to the National Highway Traffic Safety Administration (NHTSA), poor visibility was a contributing factor in over 4,000 fatal crashes in the United States in 2018.
    
    \item According to the IIHS, in foggy weather, fatal crashes happen at a rate that is 6 times higher than in clear weather.
    
    \item According to European Transport Safety Council (ETSC), over 12,000 people die on European roads each year in weather-related accidents, from highest to lowest: frost and ice, snow, and rain.
    
    \item Not only this, there are other sectors, for example, healthcare for wearable sensors, precision agriculture, and environmental monitoring, that have also seen the fruitful impact of multi-modal sensor fusion.

    \item \textbf{Healthcare sector}: wearable sensors, estimated that the global wearable device market is expected to reach over \$ 54 billion in revenue by 2027, growing at a CAGR of over 13\%
    
    \item \textbf{Precision agriculture and  environmental monitoring}: for better crop health and analyze deforestation, \$45 billion by 2026, growing at a CAGR of over 20\% 
    
    \item \textbf{Aerospace and defense}: including aircraft navigation and control, missile guidance, and military logistics. Expected to reach \$4.71 billion by 2025, at a CAGR of 8.2\%    

    \item \textbf{Industrial automation}:  increase the efficiency and productivity of manufacturing processes, as well as reduce the risk of errors and accidents
    

    
% *************************    
    % \item Who will benefit from the results of this R\&D project?
    % \item What are the benefits? Quantify the benefits with concrete numbers.
% *************************
 \end{itemize}

\section{Related Work}

\subsection{Survey of Related Work}
\begin{itemize}
    \item Bijelic et al. employed a deep learning-based transfer learning approach to address unseen adverse weather conditions
    \cite{bijelic2020seeing}
    \begin{itemize}
        \item 5 sensor modalities C-R-L-FIR-NIR
    \end{itemize}
    
    \item K-radar: 
    \cite{Paek2022Jun}
    \begin{itemize}
        \item Released 4D radar dataset
        \item Showed baseline network only, and mAP still 41.1\%
        \item But not compared with other multi-modal architectures and does not use advanced NN techniques 
    \end{itemize}
        
    
    \item C-R Fusion:
    \cite{Nobis2020May}
    \begin{itemize}
        \item Model inspired by C-L fusion
        \item Shows the importance of radar in object detection
    \end{itemize}    
    
    
    % ***********************
    % \item What have other people done to solve the problem?
    % \item You should reference and briefly discuss at least the ``top twelve'' related works
    % ***********************
\end{itemize}

\subsection{Limitation and Deficits in the State of the Art}
\begin{itemize}

    \item Most existing works fuse RGB images from visual cameras with 3D LiDAR point clouds    
    \cite{feng2020deep}
    
    \item There is no general guideline for network architecture design, and the below questions are still open\cite{Zhou2022May}: 
        \begin{itemize}
            \item “what to fuse” - lidar, radar, color camera, thermal camera, event camera, ultrasonic 
            \item “how to fuse” - addition or mean, concatenate, ensemble, mixture of experts
            \item “when to fuse” - early, mid, late, combination of all
        \end{itemize}

    \item Previous studies lack comparison with alternative models or datasets
    \item showing only results for their own baseline models and custom datasets
    
    \item None of the multi-modal sensor fusion algorithms handle temporal information
    \cite{bijelic2020seeing}
    
    \item Not much work available utilizing 4D imaging radar sensor 
    \cite{Zhou2022May}    


    % ***********************
    % \item List the deficits that you have discovered in the related work and explain them such that a person who is not deep into the technical details can still understand them.
    % For each deficit, provide at least two references
    % \item You should reference and briefly discuss at least the ``top twelve'' related works
    % ***********************
\end{itemize}

\section{Problem Statement}
\begin{itemize}
    
    \item Which of the deficits are you going to solve?
    
    \item What is your intended approach?
    \item A thorough analysis and practical implementation of state-of-the-art methods for object detection using multiple modalities including but not limited to camera, lidar, and radar
        
    \item Determining an appropriate fusion strategy to exploit the complementary characteristics of various sensors
    \begin{itemize}
        \item How to fuse camera + 4D radar data
    \end{itemize}
    
    \item Fusion of spatial and temporal information from multi-modal sensors

    \item If required, use CARLA or other simulators to validate the performance of a model 
    
    \item Conduct experiments and compare outcomes with various models and adverse weather conditions datasets
    \begin{itemize}
        \item Datasets: K-radar\cite{Paek2022Jun}, DENSE\cite{bijelic2020seeing}, aiMotive\cite{Matuszka2022Nov}
    \end{itemize}    


    \item How will you compare you approach with existing approaches?
\end{itemize}

\section{Project Plan}

\subsection{Work Packages}
\emph{Planning is the replacement of randomness by error.} (Einstein). Very much like you would never start a longer journey without a detailed travel plan, you should not start a project without a carefully though out work plan. A work package is a logical decomposition of a larger piece of work into smaller parts following a ``divide and conquer" strategy. It is very specific to the problem that you are going to address. Refrain from a rather generic decomposition. If your work plan looks similar to those of your school mates, which may address completely different problems then you have not thought carefully enough about how you approach the problem. It is ok to have two generic work packages \emph{Literature Study} and \emph{Project Report}. Discuss your work packages in the ASW seminar.

The bare minimum will include the following packages:
\begin{enumerate}
    \item[WP1] Literature Study
    \item[WP2] ...
    \item[WP3] ...
    \item  ...
    \item[WPy] Evaluation of approach and comparison with similar approaches
    \item[WPz] Project Report
\end{enumerate}

\subsection{Milestones}
Milestones mark the completion of a certain activity or at least a major achievement in an activity. Milestones are also decision points, where you reflect on what you have achieved and what options you have for continuing your work in case you have not achieved what was planned. Above all, milestones have to be measurable. As above, if your milestones are the same as those of your school mates, then you may not have thought carefully enough about how your project shall progress.
\begin{enumerate}
    \item[M1] Literature review completed and best practice identified
    \item[M2] ...
    \item[M3] ...
    \item[M4] Report submission
\end{enumerate}

\subsection{Project Schedule}
Include a Gantt chart here. It doesn't have to be detailed, but it should include the milestones you mentioned above.
Make sure to include the writing of your report throughout the whole project, not just at the end.

\begin{figure}[h!]
    \includegraphics[width=\textwidth]{images/rnd_deliverable_timeline}
    \caption{My figure caption}
    \label{fig:myfigure}
\end{figure}

\subsection{Deliverables}

\subsubsection*{Minimum Viable}
\begin{itemize}
    \item Project results required to get a satisfying or sufficient grade.
\end{itemize}

\subsubsection*{Expected}
\begin{itemize}
    \item Project results required to get a good grade.
\end{itemize}

\subsubsection*{Desired}
\begin{itemize}
    \item Project results required to get an excellent grade.
\end{itemize}

Please note that the final grade will not only depend on the results obtained in your work, but also on how you present the results.

\nocite{*}

\bibliographystyle{unsrt} % Use the plainnat bibliography style
\bibliography{bibliography.bib} % Use the bibliography.bib file as the source of references

\end{document}