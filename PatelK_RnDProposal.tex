\documentclass[rnd]{mas_proposal}
% \documentclass[thesis]{mas_proposal}

\usepackage[utf8]{inputenc}
\usepackage{amsmath}
\usepackage{amsfonts}
\usepackage{amssymb}
\usepackage{graphicx}

\title{Object detection in adverse weather conditions using tightly-coupled data-driven multimodal sensor fusion}
\author{Kevin Patel}
\supervisors{Prof. Dr.-Ing. Sebastian Houben\\M.Sc. Santosh Thoduka}
\date{February 2023}

% \thirdpartylogo{path/to/your/image}

\begin{document}

\maketitle

\pagestyle{plain}

\section{Introduction}

\subsection{Topic of This R\&D Project}
\begin{itemize}
    % *************************
    % \item Provide reasonably detailed description of what you intent to do in your R\&D project.
    % \item You may also discuss the challenges that you have to address.
    % \item Reflect on the profile of the reader and PLEAAAASE, tell a story here and refrain from bombarding the readers with details which they may not be able to appreciate.
    % **************************

    % \item What is Sensor Fusion?
    % \begin{itemize}
    %     \item The process of combining data from multiple sensors to provide a more accurate, reliable, and comprehensive understanding of an environment or situation.
    % \end{itemize}

    % Why do we need multi-modal sensor fusion?
    % Check where to put this question

    % \item To address this challenge, our research and development project aims to fuse different sensor modalities, including point cloud, pixel, and time series data, to improve perception and object detection accuracy in a variety of adverse weather conditions.

    % \item Fuse different sensor modalities like point cloud, pixels, and time series 
    % \item Improve perception in adverse weather conditions e.g., fog, rain, snow, overcast, sleet, night
    % \item Synchronization of multi-modal data
    % \item Process dense and sparse resolution sensors data
    % \item Make use of a data-driven approach
    % \item Geometrically alignments of different sensors
    % \item Real-time sensor fusion with low latency

    \item Imagine driving on a winding mountain road at night, with fog and rain obscuring your view, your vehicle's self-driving system struggles to detect objects ahead due to the challenging weather conditions. Suddenly, a deer jumps out in front of your car, causing the system to issue an alert and apply the brakes in time to avoid a collision.

    \item This scenario highlights the importance of object detection in adverse weather conditions for self-driving cars. Visual cameras, which are commonly used for object detection, may be distorted or obscured by rain, fog, snow, or low light, making it difficult to accurately detect objects on the road \cite{yurtsever2020survey} \cite{carballo2020libre} \cite{mcity2020}.

    \item To address these challenges, this project aims to implement a multimodal sensor fusion system that combines cameras, radar, and LiDAR sensors. By fusing data from multiple sensors and leveraging advanced machine learning algorithms, the goal is to enhance object detection's range, accuracy, and reliability in adverse weather conditions.

    \item The focus will also be on synchronizing multimodal data, processing dense and sparse resolution sensor data, and using a data-driven approach to optimize object detection performance.

    \item However, this project also faces several challenges. For example, different sensors may have different resolutions and sampling rates and may require sophisticated calibration and alignment techniques to ensure the accurate fusion of their data. Furthermore, processing large volumes of sensor data with minimal latency requires efficient and scalable algorithms and hardware architectures.

    \item The proposed system will be trained on a diverse dataset to ensure robustness and adaptability in different weather and lighting conditions. The system's effectiveness will be evaluated by extensive experiments and by comparing existing state-of-the-art methods.

    \item Despite the challenges, the project has the potential to revolutionize object detection in adverse weather conditions, with applications ranging from self-driving cars to surveillance and security systems. By fusing multiple sensor data sources and optimizing their fusion, situational awareness can be enhanced, enabling safer and more efficient operations in various domains.

    \item This research aims to facilitate safe and efficient self-driving in adverse weather conditions, prioritizing the safety of passengers, other drivers, and pedestrians on the road. To accomplish this, the proposed approach is to develop a sensor fusion system that operates with minimal latency, enabling data processing from multiple sensors in near real-time.

          % write code to put next item on new page
          \newpage

    \item Topic naming convention:
          \begin{itemize}
              \item Object detection
                    \begin{itemize}
                        \item Refers to the task of detecting objects within an image or video stream.
                        \item In this project, the focus is on detecting 2D objects such as cars, trucks, pedestrians, and cyclists.
                    \end{itemize}
              \item Adverse weather conditions
                    \begin{itemize}
                        \item Refers to conditions such as fog, snow, rain, overcast skies, sleet, and dust.
                        \item These conditions can make object detection more challenging due to reduced visibility or other environmental factors.
                    \end{itemize}
              \item Tightly-coupled
                    \begin{itemize}
                        \item Refers to how different modalities of data are combined and integrated at different levels.
                        \item Rather than relying solely on early, mid, or late fusion techniques, a combination of features at different levels is employed to achieve optimal fusion results.
                    \end{itemize}
              \item Data-driven
                    \begin{itemize}
                        \item Refers to the use of previously collected data or publicly available datasets to improve object detection performance.
                    \end{itemize}
              \item Multimodal
                    \begin{itemize}
                        \item Refers to the use of different data modalities to improve object detection performance.
                        \item Examples include sensors such as LiDAR, camera, IMU, GPS, infrared, and radar, with different datatypes such as point clouds, images, and time series data.
                    \end{itemize}
              \item Sensor fusion
                    \begin{itemize}
                        \item Refers to the process of fusing data from different sensors to get a better estimation of an environment and improve object detection performance.
                    \end{itemize}
          \end{itemize}

          % \begin{itemize}
          %     \item {Object detection}
          %     \begin{itemize}
          %         \item 2D object detection - car, truck, pedestrian, cycle
          %     \end{itemize}

          %     \item {Adverse weather conditions}
          %     \begin{itemize}
          %         \item Fog, snow, rainy, overcast, sleet, dust
          %     \end{itemize}

          %     \item {Tightly-coupled}
          %     \begin{itemize}
          %         \item How different modalities are combined at what level
          %         \begin{itemize}
          %             \item Eg. sensor fusion, mid fusion/feature fusion, late fusion, ROI fusion, decision fusion
          %         \end{itemize}
          %     \end{itemize}

          %     \item {Data-driven}
          %     \begin{itemize}
          %         \item Using previously collected data or publicly available datasets
          %     \end{itemize}

          %     \item {Multi-modal}
          %     \begin{itemize}
          %         \item Using different data modalities
          %         \begin{itemize}
          %             \item Sensors: LiDAR, camera, IMU, GPS, infrared, radar
          %             \item Datatypes: Point cloud, image, timer series
          %         \end{itemize}
          %     \end{itemize}

          %     \item {Sensor fusion}
          %     \begin{itemize}
          %         \item Fuse different sensors data to get a better estimation of an environment
          %     \end{itemize}
          % \end{itemize}


\end{itemize}

\subsection{Relevance of This R\&D Project}
\begin{itemize}
    % CAGR - Compound Annual Growth Rate

    \item The relevance of the research project lies in the fact that weather phenomena have a significant negative influence on traffic and transportation, which can lead to accidents, injuries, and fatalities.

    \item The statistics show that adverse weather conditions, such as rain, snow, sleet, and fog, contribute to a high number of vehicle crashes and fatalities worldwide.

    \item For example, in the United States, over 30,000 vehicle crashes occur on snowy or icy roads each year, causing over 5,000 fatalities and 418,000 injuries due to adverse weather-related crashes, according to the Federal Highway Administration (FHA) \cite{federal-highway-administration-no-date} \cite{usDepartmentofCommerce2016}.

    \item The Insurance Institute for Highway Safety (IIHS) found that in snowy weather, the fatal crash rate is 21\% higher than on clear roads, while during sleet and freezing rain, the rate is even higher at 37\%. Moreover, poor visibility is a contributing factor in over 7,000 annual crashes in the United States, according to the FHA, and in over 4,000 fatal crashes in 2018, according to the National Highway Traffic Safety Administration (NHTSA) \cite{brumbelow2022light}.

    \item In Europe, adverse weather conditions cause 25\% of all road accidents, with frost and ice, snow, and rain being the highest contributing factors, according to the European Commission and the European Transport Safety Council (ETSC). Over 12,000 people die on European roads each year in weather-related accidents \cite{cookson-2022}.

    \item Furthermore, the project's results will benefit various sectors, including autonomous vehicles, healthcare, precision agriculture, environmental monitoring, aerospace and defense, and industrial automation.

    \item The sensor fusion market for autonomous vehicles is expected to reach \$22.2 billion by 2030 at a CAGR of 25.4\%, according to Marketsandmarkets \cite{marketsandmarkets}.

    \item In the healthcare sector, wearable sensors are estimated to reach over \$1.5 billion in revenue by 2030, growing at a CAGR of over 18.3\% \cite{straitsresearch2021}.

    \item For precision agriculture and environmental monitoring, the market is expected to reach \$10.5 billion by 2026, growing at a CAGR of over 12.6\% \cite{mordorintelligence2023}.

    \item The aerospace and defense sector, including aircraft navigation and control, missile guidance, and military logistics, is expected to reach \$23.83 billion by 2027, at a CAGR of 4.21\% \cite{fortunebusinessinsights2023}.

    \item Even the industrial automation sector benefits from the sensor fusion technology as it can improve the efficiency of the production process and reduce the cost of production.


          % \item 77\% of the countries in the world receive snow \cite{Zhang2021Dec}


          % \item Weather phenomena have various negative influences on traffic and transportation. Averagely, global precipi- tation occurs 11.0\% of the time [4]. It has been proven that the risk of accident in rain conditions is 70\% higher than normal [5]. 77\% of the countries in the world re- ceive snow.

          % \item Take the United States national statistics as an example, each year over 30,000 vehicle crashes occur on snowy or icy roads or during snowfall or sleet [6], so the threat from snow is bona fide

          % \item According to Federal Highway Administration(FHA), adverse weather-related vehicle crashes cause over 5,000 fatalities and over 418,000 injuries each year in the United States. 
          % \cite{federal-highway-administration-no-date}

          % % Insurance Institute for Highway Safety (IIHS)
          % \item The IIHS also found that in snowy weather, the fatal crash rate is 21\% higher than on clear roads, while during sleet and freezing rain, the rate is even higher at 37\%.

          % \item According to the European Commission, 25\% of all road accidents in Europe happen due to adverse weather conditions, from highest to lowest: frost and ice, snow, and rain.
          % \cite{cookson-2022}

          % \item Autonomous vehicles: according to Marketsandmarkets, the sensor fusion market for autonomous vehicles is expected to reach \$ 22.2 billion by 2030 at a CAGR of 25.4\%  
          % \cite{marketsandmarkets}

          % \item According to the Federal Highway Administration (FHWA), poor visibility is a contributing factor in over 7,000 annual crashes in the United States.

          % \item According to the National Highway Traffic Safety Administration (NHTSA), poor visibility was a contributing factor in over 4,000 fatal crashes in the United States in 2018.

          % \item According to the IIHS, in foggy weather, fatal crashes happen at a rate that is 6 times higher than in clear weather.

          % \item According to European Transport Safety Council (ETSC), over 12,000 people die on European roads each year in weather-related accidents, from highest to lowest: frost and ice, snow, and rain.

          % \item Not only this, there are other sectors, for example, healthcare for wearable sensors, precision agriculture, and environmental monitoring, that have also seen the fruitful impact of multi-modal sensor fusion.

          % \item \textbf{Healthcare sector}: wearable sensors, estimated that the global wearable device market is expected to reach over \$ 54 billion in revenue by 2027, growing at a CAGR of over 13\%

          % \item \textbf{Precision agriculture and  environmental monitoring}: for better crop health and analyze deforestation, \$45 billion by 2026, growing at a CAGR of over 20\% 

          % \item \textbf{Aerospace and defense}: including aircraft navigation and control, missile guidance, and military logistics. Expected to reach \$4.71 billion by 2025, at a CAGR of 8.2\%    

          % \item \textbf{Industrial automation}:  increase the efficiency and productivity of manufacturing processes, as well as reduce the risk of errors and accidents



          % *************************    
          % \item Who will benefit from the results of this R\&D project?
          % \item What are the benefits? Quantify the benefits with concrete numbers.
          % *************************
\end{itemize}

\section{Related Work}

\subsection{Survey of Related Work}
\begin{itemize}

    % - Overall talk about:
    %     - Adverse weather
    %         - single modality based solutions
    %             - using camera
    %             - using radar
    %             - using LiDAR
    %         - multi-modality based solutions
    %             - using camera and radar
    %             - using camera and LiDAR
    %             - using radar and LiDAR
    %             - using camera, radar, and LiDAR
    %             - other misc sensors (e.g. infrared, ultrasonic, event camera, etc.)
    %     - Non adverse weather
    %         - single modality based solutions
    %             - using camera
    %             - using radar
    %             - using LiDAR
    %         - multi-modality based solutions
    %             - using camera and radar
    %             - using camera and LiDAR
    %             - using radar and LiDAR
    %             - using camera, radar, and LiDAR
    %             - other misc sensors (e.g. infrared, ultrasonic, event camera, etc.)

    % - TODO:
    %     - First, start with writing all the information about the related work
    %     - Later, we can organize it into proper sequence
    %     - Stop when you find at least 10 related papers.

    % Adverse weather influence on sensors

    \item Object detection is a fundamental computer vision problem in autonomous robots, including self-driving vehicles and autonomous drones. Such applications require 2D or 3D bounding boxes of scene objects in challenging real-world scenarios, including complex cluttered scenes, highly varying illumination, and adverse weather conditions. The most promising autonomous vehicle systems rely on redundant inputs from multiple sensor modalities \cite{caesar2020nuscenes} \cite{sun2020scalability} \cite{ziegler2014making}, including camera, LiDAR, radar, and emerging sensors such as far-infrared(FIR) and near-infrared(NIR) \cite{bijelic2020seeing}.

    \item For a typical perception system, the most common sensor is camera, and it's actually the one element that is absolutely not replaceable in autonomous driving systems. But it's also one of the most vulnerable sensors to adverse weather conditions. A camera in rain, regardless of however high resolution, can be easily incapacitated by a single water drop on the emitter or lens \cite{mardirosian2021LiDAR}. Heavy snow or hail could fluctuate the image intensity and obscure the edges of the pattern of a certain object in the image or video which leads to detection failure \cite{zang2019impact}. A particular weather phenomenon, strong light, directly from the sun or artificial light source like light pollution from a skyscraper may also cause severe trouble to cameras \cite{acarballo2020libre}.
          % (TODO: if possible add a picture of a camera in rain)

    \item Second most common sensor available on autonomous driving systems is LiDAR. For the most common weather, rain, when it’s not extreme like a normal rainy day, it doesn’t affect LiDARs that much according to the research of Fersch et al. \cite{fersch2016influence} on small aperture LiDAR sensors. More serious harm of rain happens when it becomes heavy or unbridled. Rains with a high and non-uniform precipitation rate would most likely form lumps of agglomerate fog and create fake obstacles to the LiDARs. Hasirlioglu et al. \cite{hasirlioglu2016modeling} proved that the signal reflection intensity drops significantly in a rain rate of more than 40 mm/hr. According to Zhang et al. \cite{Zhang2021Dec}, dense fog or dense smoke cause the same effect as the heavy rain. As mentioned for camera, a strong light also affect LiDAR sensors in extreme conditions \cite{acarballo2020libre}.

          % -> Now talk about radar robustness to adverse weather conditions.

    \item Radar is the third most crucial sensor in autonomous driving systems and is widely used in mass-produced cars for active safety functions, such as automatic emergency braking (AEB) and forward collision warning (FCW). However, its significance is often overlooked from the perspective of perception tasks in autonomous driving. Unlike RGB cameras that use visible light bands (384$\sim$769 THz) and LiDARs that use infrared bands (361$\sim$331 THz), Radars use relatively longer wavelength radio bands (77$\sim$81 GHz), resulting in robust measurements in adverse weathers \cite{Paek2022Jun}. As reported by Ijaz et al. \cite{ijaz2012analysis} and Ismail \cite{gultepe2008measurements}, radar exhibits lower attenuation in rainy conditions than LiDAR. The attenuation of radar at 77 GHz is approximately 3.5 times lower (10 dB/km) than that of LiDAR at 905 nm (35 dB/km), demonstrating better robustness. Multiple experiments \cite{adams2012robotic, brooker2007seeing, xu2022learned, gourova2017analysis, zang2019impact} have revealed that attenuation and backscattering under dust, fog, snow, and light rain are negligible for radar, while its performance degrades under heavy rainfall. However, one of the significant drawbacks of radar is its low resolution, which makes it difficult to use in perception tasks. The radar point cloud is much sparser than LiDAR, limiting its usability. Recently, the next generation of 4D radar has emerged, which can provide denser points compared to conventional radar sensors.
          % radar exhibits lower attenuation than LiDAR. Radar at 77 GHz demonstrates approximately 3.5 times lower attenuation (10 dB/km) compared to LiDAR at 905 nm (35 dB/km), highlighting its better robustness.

          % Summary of all 3 sensors in adverse weather conditions:
          % Adverse weather conditions, such as heavy rain, snow, and fog, can be a significant threat to safe driving. Different sensors operate in different electromagnetic wavebands, thus having different robustness to environments. A comparison of the weather effects on different sensors can be found in [217]. Visual perception is susceptible to blur, noise, and brightness distortions [218,219]. In adverse weather, LiDAR suffers from reduced detection range and blocked view in powder snow [220], heavy rain [221], and strong fog [221]. In contrast, radar is more robust under adverse weather. The effect of weather on radar can be divided into attenuation and backscattering [222]. The attenuation effect decreases the received power of the signal, and the backscattering effect increases the interference at the receiver. Experiments [43,223–225] reveal attenuation and backscattering under dust, fog, and light rain are negligible for radar, while the performance of radar degrades under heavy rainfall.


          % -> Then talk about multi-modal sensor fusion in general on KITTI dataset. But it does not have adverse weather conditions.
    \item By now, it’s almost well established that the LiDAR or Camera architecture alone is not going to navigate through adverse weather conditions with enough safety assurance. But two forces combining together would be a different story with the additional strength. As a result, groups from all over the world come up with their own permutation and combination with camera, LiDAR, radar, infrared camera, gated camera, stereo camera, weather stations and other weather-related sensors.

    \item Yang et al. \cite{yang2020radarnet} brought up a modality called RadarNet, which exploits both radar and LiDAR sensors for perception. It uses an early fusion mechanism to learn joint representations from the two sensors, and a late-fusion mechanism to exploit radar’s radial velocity evidence and improve the estimated object velocity. They validated their modality in the nuScenes dataset \cite{caesar2020nuscenes}.
          % \item Deficits or limitations of RadarNet: the radar used in nuScenes dataset has a very low resolution, and hence it's not a good choice for studying the role of radar in perception. Object detection using radars is limited by low resolution and erroneous elevation estimates \cite{ulrich2021deepreflecs} \cite{drews2022deepfusion}
          % \item Possible improvements: use a higher resolution radar sensor, and use a radar sensor with a higher elevation angle. Like the radar used in K-Radar dataset, which is a 4D radar with elevation angle of 30 degrees.

    \item Liu et al. \cite{liu2021robust} raised a robust target recognition and tracking method combining radar and camera information under severe weather conditions, with radar being the main hardware and camera the auxiliary. They tested their scheme in rain and fog including night conditions when visibility was the worst. Results show that radar has a pretty high accuracy in detecting moving targets in wet weather, while the camera is better at categorizing targets and the combination beats LiDAR alone detection by over a third.
          % \item Deficits or limitations of Liu et al. cite{liu2021robust}: NEED TO Check
          % \item possible improvements: NEED TO Check

    \item FLIR System Inc. \cite{fused_aeb} and the VSI Labs \cite{VSILabs} tested the world’s first fused automated emergency braking (AEB) sensor suite in 2020, equipped with a thermal long-wave infrared (LWIR) camera, a radar and a visible camera. LWIR covers the wavelength ranging from 8 µm to 14 µm and such camera operates under ambient temperature known as the uncooled thermal camera. This sensor suite was tested along with several cars with various AEB features employing radar and visible camera against day-time, nighttime and tunnel exit into sun glare. The comparison showed that although most AEB systems work fine in the daytime, normal AEB almost hit every mannequin under those adverse conditions, while the LWIR sensor suite never knocked down a single one. This work shows the potential of the camera and radar fusion in adverse weather conditions.
          % \item Deficits or limitations of FLIR System Inc. \cite{fused_aeb} and the VSI Labs \cite{VSILabs}: The durability of such temperature-sensitive devices needs further validation in real environments in the future to ensure their usefulness in adverse weather conditions. 
          % \item possible improvements: NEED TO Check

    \item Radecki et al. \cite{radecki2016all} extensively summarized the performance of each sensor against all kinds of weather including wet conditions, day \& night, cloudy, glare, and dust. They formulated a system with the ability of tracking and classification based on the probability of joint data association. Their vision detection algorithm is realized by using sensor subsets corresponding to various weather conditions with realtime joint probabilistic perception. The essence of such fusion is about real-time strategy shift. Sensor diversity improves the perception ability general lower bound, but the intelligent choice of sensor weighting and accurately quantified parameters based on the particular weather determine the ceiling of the robustness and reliability of such modalities.
          % \item Deficits or limitations of Radecki et al. \cite{radecki2016all}: The majority of published studies are conducted under optimal weather conditions, with no guarantee of robustness \cite{emzivat2018formal}. Not explored the performance in heavy-traffic or urban areas. Not explored the deep learning based fusion techniques. 
          % \item possible improvements: use the Similarly method or techniques to detect objects in K-Radar dataset.

    \item Bijelic et al. \cite{bijelic2020seeing} from Mercedes-Benz AG conducted a study on improving detection performance in adverse weather conditions using a deep multimodal sensor fusion approach. The authors equipped their test vehicle with various sensors, including stereo RGB cameras, a NIR camera, a 77 GHz radar, two LiDARs, an FIR camera, a weather station, and a road-friction sensor. They proposed an entropy-steered fusion approach where regions with low entropy were attenuated while entropy-rich regions were amplified during feature extraction. The exteroceptive sensor data were concatenated and trained using clear weather data, demonstrating strong adaptation to unseen adverse weather data. The fusion network was designed to generalize across different scenarios, and all the sensor data were projected into the camera coordinate system to ensure consistency. The fused detection performance outperformed LiDAR or image-only approaches under fog conditions.
          % \item TODO: add images
          % \item Deficits or limitations of Bijelic et al. \cite{bijelic2020seeing}:One of the problems with radar signal transformations exploited by all of the methods mentioned above is that some essential radar information could be lost while conducting the projections. Besides, some of the spatial information from the original radar signals could not be utilized. The blemish in this modality is that the amount of sensors exceeds the normal expectation of an ADS system. More sensors require more power supply and connection channels which is a burden to the vehicle itself and proprietary weather sensors are not exactly cost-friendly. Even though such an algorithm is still real-time processed, given the bulk amount of data from multiple sensors, the response and reaction time becomes something that should be worried about \cite{Zhang2021Dec}
          % \item possible improvements: The network architecture can be improved by using transformer-based architectures. The author mentioned about the robustness of radar in adverse weather conditions, but the performance of the radar is only limited by the low azimuth and elevation resolution. The performance of the radar can be improved by using a 4D radar, which has a higher resolution compared to the 2D radar used in this work.

    \item Bijelic et al. \cite{bijelic2020seeing} also provided the SeeingThroughFog dataset for further research on multimodal sensor fusion in adverse weather conditions. This dataset comprises 10,000 km of driving data in Northern Europe, recorded during February and December 2019, under varying weather and illumination conditions. The dataset includes annotations for 5.5 k clear weather frames, 1 k dense fog frames, 1 k light fog frames, and 4 k frames captured in snow/rain.

    \item Qian et al. \cite{qian2021robust} introduced a Multimodal Vehicle Detection Network (MVDNet) featuring LiDAR and radar. It first extracts features and generates proposals from both sensors, and then the multimodal fusion processes region-wise features to improve detection. They created their own training dataset based on the Oxford Radar Robotcar \cite{barnes2020oxford} and the evaluation shows much better performance than LiDAR alone in fog conditions.
          % \item Deficits or limitations of Qian et al. \cite{qian2021robust}: The dataset used for evaluation is limited to a specific scenario, which may not be representative of all real-world situations. The misalignment between lidar and radar data is not corrected in the dataset, which affects the performance of MVDNet. The study didn't compare the performance of MVDNet with other other camera and radar fusion methods.
          % \item possible improvements: the proposed architecture can be tested with other modalities such as thermal camera or rgb camera.

    \item Rawashdeh et al. \cite{rawashdeh2021drivable} include cameras, LiDAR and radar in their CNN sensor fusion for drivable path detection, and used DENSE \cite{bijelic2020seeing} dataset. This multi-stream encoder-decoder almost complements the asymmetrical degradation of sensor inputs at the largest level. The depth and the number of blocks of each sensor in the architecture are decided by their input data density, of which camera has the most, LiDAR the second and radar the last, and the outputs of the fully connected network are reshaped into a 2-D array which will be fed to the decoder. Their model can successfully ignore the lines and edges that appeared on the road which could lead to false interpretation and delineate the general drivable area.
          % \item Deficits or limitations of Rawashdeh et al. \cite{rawashdeh2021drivable}: Such fusion modality certainly can do more than countering snow conditions but other low-visibility scenarios such as fog, rain, and dust. The paper does not compare the proposed algorithm with other state-of-the-art methods for drivable path detection in poor weather conditions.

          % TODO: check the need of this point, as it's not about multi-modal fusion, it's only about adverse weather conditions
    \item There are studies out there that use de-hazing techniques to remove the bad effects fro adverse weather. While physical priors were previously used \cite{tan2008visibility} \cite{tarel2009fast}, data-driven methods using deep learning have been introduced. However, deep de-hazing models have high computational complexity and are unsuitable for ultra-high-definition images. Chen et al. \cite{chen2021psd} found that models trained on synthetic images do not generalize well to real-world hazy images, while Zhang et al. \cite{zhang2021learning} used temporal redundancy to perform video de-hazing and collected a dataset of real-world hazy and haze-free videos. Although collecting pairs of hazy and haze-free ground-truth images is challenging, professional haze/fog generators exist to simulate real-world conditions \cite{musat2021multi} \cite{timofte2018ntire}.

    \item The rapid developments of autonomous driving especially in adverse weather conditions benefit a lot from the availability of simulation platforms and experimental facilities like fog chambers or test roads. Virtual platforms like the well-known CARLA \cite{dosovitskiy2017carla} simulator, enable researchers to construct custom-designed complex road environments and non-ego participants with infinite scenarios where it would be extremely hard and costly in real field experiments. Moreover, for weather conditions, the appearing of each kind of weather especially season-related or extreme climates related is not on call at all times. For example, it’s impossible for tropical areas to have the opportunity to do snow tests; and natural rain showers might not be long enough to collect experimental data. Most importantly, adverse conditions are usually dangerous for driving and subjects always face safety threats in normal field tests, while absolute zero risks are something that simulators can guarantee \cite{Zhang2021Dec}.

          % -> somewhere talk about multimodal sensor fusion on synthetic data.
    \item Few researchers have explored synthetic data generation for adverse weather conditions using GAN-based techniques from clean weather dataset eg. KITTI \cite{geiger2012we}, Cityscapes \cite{cordts2016cityscapes}, etc \cite{sun2021multi} \cite{zheng2020forkgan} \cite{lee2022perception}.
          % \item Deficits or limitations of the synthetic data generation for adverse weather: these methods are mainly evaluated on rendered synthetic fog/rain images and a few real images assuming specific fog/rain models. It is thus unclear how these algorithms would be proceeding on various adverse weather conditions and how the progress could be measured in the wild \cite{hassaballah2020vehicle}.

          % -> so some new dataset required. talk about some new datasets with adverse weather conditions.
    \item Most of the deep multimodal perception methods are based on supervised learning. Therefore, multimodal datasets with labeled ground-truth are required for training such deep neural networks. There are several multimodal datasets available, however, most of these datasets are collected under clear weather conditions. The datasets collected under adverse weather conditions are limited. The following are some of the multimodal datasets for testing the performance of deep multimodal perception methods in adverse weather conditions. Out of all the datasets, only a recently released K-Radar \cite{Paek2022Jun} has the 4D radar sensor. Here C-R-L-N-F refers to Camera, Radar, LiDAR, and Near-infrared amd Far-infrared sensors, respectively.


          \begin{table}[h]
              \centering
              \caption{List multimodal datasets with adverse weather conditions}
              \label{tab:my-table}
              \begin{tabular}{|l|l|l|l|}
                  \hline
                  \textbf{Name}       & \textbf{Sensors} & \textbf{Link}               & \textbf{Year} \\ \hline
                  DENSE               & CRLNF            & \cite{bijelic2020seeing}    & 2020          \\ \hline
                  EU Long-term        & CRL              & \cite{yan2020eu}            & 2020          \\ \hline
                  nuScenes            & CRL              & \cite{caesar2020nuscenes}   & 2020          \\ \hline
                  The Oxford RobotCar & CRL              & \cite{barnes2020oxford}     & 2020          \\ \hline
                  RADIATE             & CRL              & \cite{sheeny2021radiate}    & 2021          \\ \hline
                  K-Radar             & CRL              & \cite{Paek2022Jun}            & 2022          \\ \hline
                  aiMotive            & CRL              & \cite{matuszka2022aimotive} & 2022          \\ \hline
                  Boreas              & CRL              & \cite{burnett2022boreas}    & 2022          \\ \hline
                  WADS                & CRLNF            & \cite{kurup2022winter}      & 2023          \\ \hline
              \end{tabular}
          \end{table}

          % -> talk about multimodal sensor fusion on these new datasets.
          % -> more recent work towards multimodal sensor fusion on real data with adverse weather conditions.

          % #####
          % If so many papers are there, then only include with the adverse weather conditions

    \item To address the problem of when to fuse the data in the neural network architecture, Nobis et al. \cite{nobis2019deep} proposed a CameraRadarFusionNet (CRF-Net), which was inspired from camera-LiDAR fusion \cite{yu2019multi} and \cite{caltagirone2019lidar}, to learn at which level the fusion of the sensor data was the most beneficial for the detection task. They used nuScenes \cite{caesar2020nuscenes} dataset and released their own TUM dataset. Furthermore, they introduced a new training strategy to focus the learning on a specific sensor type, which was called BlackIn. For feature fusion, the element-wise addition was adopted as the fusion operation. Their fusion method outperformed the image-only network on both datasets, which again shows the importance of fusing radar data into the detection task.
    % \item TODO: add images
    % \item Deficits or limitations of the Nobis et al. \cite{nobis2019deep} method: the baseline image network was 43.47\% on average precision, while the CRF-Net was 43.95\% on average precision. In conclusion, the improvement of detection performance was limited and the feature fusion method was very simple \cite{chang2020spatial}. According to Chang et al. \cite{safa2021fail}, the proposed study does not provide an RGB sensor ablation study and it is therefore unclear whether their system is robust towards a hard camera failure. Also the performance could be improved with pre-processing radar data before fusion.


    % \item K-radar:
    % \cite{Paek2022Jun}
    % \begin{itemize}
    %     \item Released 4D radar dataset
    %     \item Showed baseline network only, and mAP still 41.1\%
    %     \item But not compared with other multi-modal architectures and does not use advanced NN techniques
    % \end{itemize}

          %   Problem with Camera and LiDAR sensors: 
          % (reference: https://www.semanticscholar.org/paper/RADIATE%3A-A-Radar-Dataset-for-Automotive-Perception-Sheeny-Pellegrin/fde35ee9b345265fc26339148d440e1139ae10ae)
          %       - Camera and LiDAR are the two primary perceptual sensors that are usually adopted. However, since they are visible spectrum sensors, their data is affected dramatically by bad weather conditions, causing attenuation, multiple scattering and turbulence [4]–[8].
          %       - [4] M. Pfennigbauer, C. Wolf, J. Weinkopf, and A. Ullrich, “Online waveform processing for demanding target situations,” in Laser Radar Technology and Applications XIX; and Atmospheric Propagation XI, vol. 9080. International Society for Optics and Photonics, 2014, p. 90800J. 
          % [5] C. Sakaridis, D. Dai, and L. Van Gool, “Semantic foggy scene understanding with synthetic data,” International Journal of Computer Vision, vol. 126, no. 9, pp. 973–992, Sep 2018. [Online]. Available: https://doi.org/10.1007/s11263-018-1072-8 
          % [6] M. Bijelic, T. Gruber, and W. Ritter, “A benchmark for LiDAR sensors in fog: Is detection breaking down?” in 2018 IEEE Intelligent Vehicles Symposium (IV). IEEE, 2018, pp. 760–767. 
          % [7] M. Kutila, P. Pyykonen, H. Holzh ¨ uter, M. Colomb, and P. Duthon, ¨ “Automotive LiDAR performance verification in fog and rain,” in 2018 21st International Conference on Intelligent Transportation Systems (ITSC). IEEE, 2018, pp. 1695–1701. 
          % [8] A. M. Wallace, A. Halimi, and G. S. Buller, “Full waveform LiDAR for adverse weather conditions,” IEEE Transactions on Vehicular Technology, vol. Early Access, 2020. [Online]. Available: https://ieeexplore.ieee.org/document/9076331

          % - LiDAR sensor:
          % - Limitation: 
          %     - SeeingThroughFog
          %     - C. Goodin, D. Carruth, M. Doude, and C. Hudson. Predicting
          %     the influence of rain on LiDAR in adas. Electronics, 8, 2019.
          %     - R. Heinzler, F. Piewak, P. Schindler, and W. Stork. Cnnbased LiDAR point cloud de-noising in adverse weather. IEEE
          %     Robotics and Automation Letters, 5(2):2514–2521, 2020.
          %     - R. H. Rasshofer, M. Spies, and H. Spies. Influences of
          %     weather phenomena on automotive laser radar systems. Advances in Radio Science, 9:49–60, 2011.


          % Radar sensor:
          % - On the other hand, a radar sensor is known to be more robust in adverse weather conditions [9]–[11].
          %     - [9] L. Daniel, D. Phippen, E. Hoare, A. Stove, M. Cherniakov, and M. Gashinova, “Low-thz radar, LiDAR and optical imaging through artificially generated fog,” in International Conference on Radar Systems (Radar 2017). IET, 2017, pp. 1–4. 
          %     - [10] F. Norouzian, E. Marchetti, E. Hoare, M. Gashinova, C. Constantinou, P. Gardner, and M. Cherniakov, “Experimental study on low-thz automotive radar signal attenuation during snowfall,” IET Radar, Sonar & Navigation, vol. 13, no. 9, pp. 1421–1427, 2019.
          %     - [11] F. Norouzian, E. Marchetti, M. Gashinova, E. Hoare, C. Constantinou, P. Gardner, and M. Cherniakov, “Rain attenuation at millimeter wave and low-thz frequencies,” IEEE Transactions on Antennas and Propagation, vol. 68, no. 1, pp. 421–431, 2019.
          % \end{itemize}


          % ############################
          % Below points are random and not in order:

          % -> I can add something like about camera:
          %     - There are many camera only object detection methods available like YOLO, SSD, Faster R-CNN, etc. but they are not robust to adverse weather conditions.
          % -> When writing about datasets, start with:
          %     Most deep multi-modal perception methods are based on supervised learning. Therefore, multi-modal datasets with labeled ground-truth are required for training such deep neural networks.
          %     Some of the datasets are generated synthetically or rendered from real-world data. However, these methods are mainly evaluated on rendered synthetic fog/rain images   and a few real images assuming specific fog/rain models. It is thus unclear how these algorithms would be proceeding on various adverse weather conditions and how the progress could be measured in the wild.
          %   Modelling of radar sensor is not an easy task. It is not easily possible to model the radar sensor in a way that it can be used in a real-world scenario.
          %     Recently a few datasets have released multi-modal dataset including radar sensor in addition to camera and LiDAR. And out of these, very few datasets are available with adverse weather conditions.
          % -> What to fuse? - talks about different sensors and their limitations
          % -> How to fuse? - talks about different data fusion methods
          % -> When to fuse? - talks about at what stage of the network to fuse the data

          % #########################################################
          % REMEMBER< HAVE TO SEND THE DRAFT TO PROF. ON 12th of APRIL !!!!!!!!!!!!!!!!!!!!!!
          % #########################################################

          % ***********************
          % \item What have other people done to solve the problem?
          % \item You should reference and briefly discuss at least the ``top twelve'' related works
          % ***********************
\end{itemize}

\subsection{Limitation and Deficits in the State of the Art}
\begin{itemize}

    % #######################################
    % Start writing limitations
    % #######################################

    \item Most existing works fuse RGB images from visual cameras with 3D LiDAR point clouds
          \cite{feng2020deep}

    \item There is no general guideline for network architecture design, and the below questions are still open\cite{Zhou2022May}:
          \begin{itemize}
              \item “what to fuse” - LiDAR, radar, color camera, thermal camera, event camera, ultrasonic
              \item “how to fuse” - addition or mean, concatenate, ensemble, mixture of experts
              \item “when to fuse” - early, mid, late, combination of all
          \end{itemize}

    \item Previous studies lack comparison with alternative models or datasets
    \item showing only results for their own baseline models and custom datasets

    \item None of the multi-modal sensor fusion algorithms handle temporal information
          \cite{bijelic2020seeing}

    \item Not much work available utilizing 4D imaging radar sensor
          \cite{Zhou2022May}

    \item Most of the recently published multimodal datasets are releasing baseline models with very simple fusion methods. That can be improved with more advanced fusion methods transformer-based, gated fusion, etc.


          %======= Content ===================
          % Multi-modal Sensor Fusion for Auto Driving Perception: A Survey
          % - Current fusion models suffer from the problem of misalignment and information loss [13, 67, 98]. Besides, the flat fusion operations [20, 76] also prevent the further improvement of perception task performance. We summarize them as two aspects: Misalignment and Information Loss, More Reasonable Fusion Operations.
          % - More Advanced Fusion Methodology
          %     - Misalignment and Information Loss
          %     - More Reasonable Fusion Operations
          % - Multi-Source Information Leverage
          % - With More Potential Useful Information
          % - Self-Supervision for Representation Learning
          % - Intrinsic Problems in Perception Sensors
          % - Data Domain Bias
          % - Conflicts with Data Resolution



          % ***********************
          % \item List the deficits that you have discovered in the related work and explain them such that a person who is not deep into the technical details can still understand them.
          % For each deficit, provide at least two references
          % \item You should reference and briefly discuss at least the ``top twelve'' related works
          % ***********************
\end{itemize}

\section{Problem Statement}
\begin{itemize}

    % \item Which of the deficits are you going to solve?

    % \item What is your intended approach?

    % \item How will you compare you approach with existing approaches?

    \item A thorough analysis and practical implementation of state-of-the-art methods for object detection using multiple modalities including but not limited to camera, LiDAR, and radar

    \item Determining an appropriate fusion strategy to exploit the complementary characteristics of various sensors
          \begin{itemize}
              \item How to fuse camera + 4D radar data
          \end{itemize}

    \item Fusion of spatial and temporal information from multimodal sensors

    \item If required, use CARLA or other simulators to validate the performance of a model

    \item Conduct experiments and compare outcomes with various models and adverse weather conditions datasets
          \begin{itemize}
              \item Datasets: K-radar\cite{Paek2022Jun}, DENSE\cite{bijelic2020seeing}, aiMotive\cite{Matuszka2022Nov}
          \end{itemize}



\end{itemize}

\section{Project Plan}

\subsection{Work Packages}
\emph{Planning is the replacement of randomness by error.} (Einstein). Very much like you would never start a longer journey without a detailed travel plan, you should not start a project without a carefully though out work plan. A work package is a logical decomposition of a larger piece of work into smaller parts following a ``divide and conquer" strategy. It is very specific to the problem that you are going to address. Refrain from a rather generic decomposition. If your work plan looks similar to those of your school mates, which may address completely different problems then you have not thought carefully enough about how you approach the problem. It is ok to have two generic work packages \emph{Literature Study} and \emph{Project Report}. Discuss your work packages in the ASW seminar.

The bare minimum will include the following packages:
\begin{enumerate}
    \item[WP1] Literature Study
    \item[WP2] ...
    \item[WP3] ...
    \item  ...
    \item[WPy] Evaluation of approach and comparison with similar approaches
    \item[WPz] Project Report
\end{enumerate}

\subsection{Milestones}
Milestones mark the completion of a certain activity or at least a major achievement in an activity. Milestones are also decision points, where you reflect on what you have achieved and what options you have for continuing your work in case you have not achieved what was planned. Above all, milestones have to be measurable. As above, if your milestones are the same as those of your school mates, then you may not have thought carefully enough about how your project shall progress.
\begin{enumerate}
    \item[M1] Literature review completed and best practice identified
    \item[M2] ...
    \item[M3] ...
    \item[M4] Report submission
\end{enumerate}

\subsection{Project Schedule}
Include a Gantt chart here. It doesn't have to be detailed, but it should include the milestones you mentioned above.
Make sure to include the writing of your report throughout the whole project, not just at the end.

\begin{figure}[h!]
    \includegraphics[width=\textwidth]{images/rnd_deliverable_timeline}
    \caption{My figure caption}
    \label{fig:myfigure}
\end{figure}

\subsection{Deliverables}

\subsubsection*{Minimum Viable}
\begin{itemize}
    % \item Project results required to get a satisfying or sufficient grade.
    \item Comparative analysis of two methods on two datasets
    \item Detecting single class of objects ex. car, pedestrian, cyclist, truck etc.

\end{itemize}

\subsubsection*{Expected}
\begin{itemize}
    % \item Project results required to get a good grade.
    \item Compare more advance methods with baseline methods on different datasets
    \item Detecting multi classes of objects ex. car, pedestrian, cyclist, truck etc.
\end{itemize}

\subsubsection*{Desired}
\begin{itemize}
    % \item Project results required to get an excellent grade.
    \item Run experiments on CARLA simulator to validate the performance of a model
          \begin{itemize}
              \item Note: CARLA simulator doesn't support 4D radar sensor
          \end{itemize}
    \item Utilizing spatial and temporal information from multimodal sensors

          % ############################
          % Future Work
          % \item Object tracking
          % \item 3D object detection
          % ############################

\end{itemize}

Please note that the final grade will not only depend on the results obtained in your work, but also on how you present the results.

\nocite{*}

\bibliographystyle{unsrt} % Use the plainnat bibliography style
\bibliography{bibliography.bib} % Use the bibliography.bib file as the source of references

\end{document}